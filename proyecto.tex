\documentclass[conference]{IEEEtran}
\IEEEoverridecommandlockouts
% The preceding line is only needed to identify funding in the first footnote. If that is unneeded, please comment it out.
\usepackage{cite}
\usepackage{amsmath,amssymb,amsfonts}
\usepackage{algorithmic}
\usepackage[spanish]{babel}
\usepackage[utf8]{inputenc}
\usepackage{graphicx}
\usepackage{textcomp}
\usepackage{xcolor}
\def\BibTeX{{\rm B\kern-.05em{\sc i\kern-.025em b}\kern-.08em
    T\kern-.1667em\lower.7ex\hbox{E}\kern-.125emX}}
\begin{document}

\title{
{\large \textsc{Escuela Superior Politécnica del Litoral} \\}
Implementación de Servicio de
Telefonía IP (VoIP)\\
{\normalsize Administración de Sistemas Operativos - CCPG1031: Proyecto Final 
}}

\author{\IEEEauthorblockN{1\textbf{Lino Ontano}}
\IEEEauthorblockA{	\textit{Telemática}
\\Guayaquil - Ecuador \\
lontano@espol.edu.ec
}
\and
\IEEEauthorblockN{2\textbf{Lino Ontano}}
\IEEEauthorblockA{	\textit{Telemática}
\\Guayaquil - Ecuador \\
lontano@espol.edu.ec
}
\and
\IEEEauthorblockN{2\textbf{Lino Ontano}}
\IEEEauthorblockA{	\textit{Telemática}
\\Guayaquil - Ecuador \\
lontano@espol.edu.ec
}
}


\maketitle
\begin{abstract}
	Esta investigación, permitirá conocer más a fondo las plataformas de ``Prototyping" Raspberry Pi 3, Arduino uno R3, Beaglebone black y pcDuino.
\end{abstract}

\begin{IEEEkeywords}
	component, formatting, style, styling, insert
\end{IEEEkeywords}
\section{Introducción}\label{sec:int}
El internet de las cosas (IoT) se está desarrollando a un ritmo rápido, gracias en parte a una explosión en la disponibilidad de hardware de computación pequeño y barato. Los kits de creación de prototipos IoT y las placas de desarrollo combinan microcontroladores y procesadores con chips inalámbricos y otros componentes en un paquete preconfigurado, listo para el programa.

Todos estos dispositivos conforman parte de las nuevas plataformas de prototipado que se utilizan hoy en día. Dado a su tamaño y precio, resulta muy cómodo su utilización actualmente, y de suma importancia saber cómo funcionan.


\section{Antecedentes}\label{sec:ant}
\begin{enumerate}
\item\textbf{ Prototipo:} es un modelo (representación, demostración o simulación) fácilmente ampliable y modificable de un sistema planificado, probablemente incluyendo su interfaz y su funcionalidad de entradas y salidas.
\item \textbf{Microcontrolador:} es un circuito integrado programable, capaz de ejecutar las órdenes grabadas en su memoria.
\item \textbf{SBC:} una placa computadora u ordenador de placa reducida (en inglés: Single Board Computer) es una computadora completa en un sólo circuito. El diseño se centra en un sólo microprocesador con la RAM, I/O y todas las demás características de un computador funcional en una sola tarjeta que suele ser de tamaño reducido.
\item \textbf{Microprocesador:} es el circuito integrado central más complejo de un sistema informático, encargado de ejecutar los programas, desde el sistema operativo hasta las aplicaciones de usuario.
\end{enumerate}

\section{Plataformas de ``Prototyping"}\label{sec:esc}
Se investigó las siguientes plataformas de ``Prototyping" donde se analizó su precio en el mercado, sistemas operativos que utilizan y demás características que nos permita adentrarnos en este mundo del prototipado por medio de SBC.
\subsection{Raspberry Pi 3}
Es el modelo más antiguo de la tercera generación de Raspberry Pi. Reemplazó el Raspberry Pi 2 Model B en febrero de 2016. Desarrollado en el Reino Unido por la fundación \textbf{Raspberry Pi}, con el objetivo de estimular la enseñanza de informática en las escuelas. \\
Salió a la luz en el año 2016, renueva procesador, una vez más de la compañía Broadcom, una vez más un Quad-Core, pero pasa de 900 MHz a 1.20 GHz. Mantiene la RAM de 1GB. Su novedad fue la inclusión de Wi-Fi y Bluetooth (4.1 Low Energy) sin necesidad de adaptadores.
\begin{figure}[h]
	%\centerline{\includegraphics[width=0.3\textwidth]{platform/rasp.jpg}}
	\caption{Raspberry Pi3 Modelo B.}
	\label{fig:ant01}
\end{figure}
Según Amazon, el precio está oscilando entre 50 y 70 dólares, dependiento del kit a comprar, donde incluyen módulos y memorias adicionales para su uso con la Raspberry.\\
Las características principales se las puede observar en el cuadro \ref{tab:rb01}.
\begin{table}[tbp]
\begin{center}
	\begin{tabular}{|p{2.5cm}|p{5.5cm}|}
	\hline 
	\textbf{Especificaciones} &\textbf{Raspberry Pi3 Modelo B} \\ \hline
	CPU  &1.2GHz 64-bit quad-core ARMv8 \\\hline
	Memoria (SDRAM) &1 GB compartidos con la GPU \\\hline
	Puertos USB &4 \\\hline
	Almacenamiento integrado &MicroSD \\\hline
	Conectividad de red &10/100 Ethernet RJ-45 vía hub USB, Wifi 802.11n, Bluetooth 4.1 \\\hline
	Fuente de alimentación &5 V vía micro USB o GPIO header \\\hline
	Sistemas Operativos Soportados &GNU/Linux: Debian (Raspbian), Fedora (Pidora), Arch Linux (Arch Linux ARM), Slackware Linux, SUSE Linux Enterprise Server for ARM. RISC OS.\\\hline
\end{tabular}\vspace{0.25cm}
\caption{Características Raspeberry Pi 3 Modelo B}
\label{tab:rb01}
\end{center}
\end{table}

	\subsection{Arduino Uno R3}
	El Arduino Uno R3 utiliza el microcontrolador ATmega328. En adición a todas las características de las tarjetas anteriores, el Arduino Uno utiliza el ATmega16U2 para el manejo de USB en lugar del 8U2 (o del FTDI encontrado en generaciones previas). Esto permite ratios de transferencia más rápidos y más memoria. No se necesitan drivers para Linux o Mac (el archivo inf para Windows es necesario y está incluido en el IDE de Arduino).\\
	\textbf{Arduino} es una compañía open source y open hardware, así como un proyecto y comunidad internacional que diseña y manufactura placas de desarrollo de hardware para construir dispositivos digitales y dispositivos interactivos que puedan sensar y controlar objetos del mundo real. \\
	El microcontrolador de la placa se programa usando el “Arduino Programming Language” (basado en Wiring) y el “Arduino Development Environment” (basado en Processing). Los proyectos de Arduino pueden ser autonomos o se pueden comunicar con software en ejecución en un ordenador (por ejemplo con Flash, Processing, MaxMSP, etc.).
	\begin{figure}[h]
	%\centerline{\includegraphics[width=0.25\textwidth]{platform/ardui}}
	\caption{Arduino Uno R3.}
	\label{fig:ardui}
	\end{figure}
	 El precio en Amazon oscila entre los 12 y 30 dólares, dependiendo del kit a comprar.\\
	Las características principales están presentes en el cuadro \ref{tab:ardui01}.
	\begin{table}[h]
\begin{center}
	\begin{tabular}{|p{2.5cm}|p{5.5cm}|}
	\hline 
	\textbf{Especificaciones} &\textbf{Arduino Uno R3} \\ \hline
	Microcontrolador  &ATmega328 \\\hline
	Voltaje de entrada &7 - 12 V \\\hline
	Entradas digitales &14 pines digitales de I/0 (6 salidas PWM) \\\hline
	Entradas analógicas & 6 entradas análogas \\\hline
	Memoria Flash &32k\\\hline
	Reloj &16MHz \\\hline
	Sistemas Operativos Soportados &N/A.\\\hline
\end{tabular}\vspace{0.25cm}
\caption{Características Arduino Uno R3}
\label{tab:ardui01}
\end{center}
\end{table}
	\subsection{Beaglebone black}
	Es una computadora barebone de desarrollo, la sucesora de la beaglebone lanzada en octubre de 2011. El precio está en 45 dólares y entre otras diferencias incrementa la RAM a 512 MB, el reloj de procesador a 1GHz, y añade HDMI y 2 GB de memoria flash eMMC. También se entrega con kernel Linux 3.8, permitiendo a la BeagleBone Black tener la ventaja del Gestor de Renderizado Directo (DRM).Demás especificaciones se observa en el cuadro \ref{tab:bb01}
		\begin{table}[h]
\begin{center}
	\begin{tabular}{|p{2.5cm}|p{5.5cm}|}
	\hline 
	\textbf{Especificaciones} &\textbf{BeagleBone Black} \\ \hline
	CPU  &Cortex-A8 + 2xPRU (200MHz) \\\hline
	Frecuencia SoC &1000MHz \\\hline
	DSP &DDR3 \\\hline
	Memoria & 512 \\\hline
	Memoria Flash &32k\\\hline
	Reloj &16MHz \\\hline
	Sistemas Operativos Soportados &Rowboat,Angstrom, Fedora, FreeBSD, MINIX 3, NetBSD, OpenBSD, openSUSE, QNX, RISC OS, Ubuntu, Void Linux, Windows Embedded.\\\hline
\end{tabular}\vspace{0.25cm}
\caption{Características BeagleBone Black}
\label{tab:bb01}
\end{center}
\end{table}
\begin{figure}[h]
	%\centerline{\includegraphics[width=0.2\textwidth]{platform/bb.jpg}}
	\caption{BeagleBone Black.}
	\label{fig:ant01}
\end{figure}
\subsection{pcDuino}
	\textbf{pcDuino} es una mini computadora o plataforma de computadora de una sola placa que funciona con PC como el sistema operativo, como Ubuntu y Android ICS. Da salida a la pantalla a HDMI. Además, tiene una interfaz de encabezados de hardware compatible con Arduino (TM). pcDuino puede usarse para enseñar Python, C y más cosas interesantes.\\
	\begin{figure}[h]
	%\centerline{\includegraphics[width=0.2\textwidth]{platform/pcd.png}}
	\caption{pcDuino1.}
	\label{fig:ant01}
\end{figure}
	\textit{pcDuino1} es una plataforma de mini PC rentable y de alto rendimiento que ejecuta PC como SO, como Ubuntu y Android ICS. Envía su pantalla a un televisor o monitor habilitado con HDMI a través de la interfaz HDMI incorporada. Está especialmente dirigido a las crecientes demandas de la comunidad de código abierto. La plataforma podría funcionar como un sistema operativo tipo PC completo con una cadena de herramientas fácil de usar y compatible con el popular ecosistema Arduino, como Arduino Shields (puede que necesite un puente protector) y proyectos de código abierto, etc. Las características principales se las observa en el cuadro \ref{tab:pcd01}.\\
			\begin{table}[h]
\begin{center}
	\begin{tabular}{|p{2.5cm}|p{5.5cm}|}
	\hline 
	\textbf{Especificaciones} &\textbf{pcDuino1} \\ \hline
	CPU  &1GHz ARM Cortex A8\\\hline
	DRAM &1 GB \\\hline
	Almacenamiento OnBoard &2GB Flash, microSD card slot hasta 32 GB \\\hline
	Interfaz de Red & 10/100Mbps RJ45 y USB Wifi extensión\\\hline
	Alimentación &5V, 2A\\\hline
	Sistemas Operativos Soportados &Linux3.0 + Ubuntu 12.04Android ICS 4.0\\\hline
\end{tabular}\vspace{0.25cm}
\caption{Características pcDuino1}
\label{tab:pcd01}
\end{center}
\end{table}
\section{Comparación de Plataformas}
Cada uno de estas mini computadoras tienen su correcto desarrollo dependiendo del aplicativo en el que le estén usando, es decir su uso dependerá de la función que lo pongas a realizar. La de menor escalabilidad podríamos mencionar a Arduino, ya que en primer lugar es un microcontrolador, y hay cosas en las que se queda corto, pero es ideal para llevar control de todo tipo de sensores y actadores y por su precio más todavía. Los restantes dependerá del sistema operativo a usarse, y los GPIOS que requeramos usar, si queremos un sistema operativo en tiempo real, la tasa de datos que queramos procesar, todas esas consideraciones hay que tener en cuenta al momento de escoger que plataforma utilizar.
\section{Conclusiones}
\begin{itemize}
	\item Las plataformas de prototipado permite elaborar a bajo costo operaciones de hardware por medio de periféricos de manera más sencilla.
	\item La utilización de las plataformas dependerá de la función que quiera realizar.
\end{itemize}
 
\begin{thebibliography}{00}
\bibitem{b1}  Liz upton, ``Introducing the New Out Of Box Software (NOOBS)'' , Raspberry Pi, 2013.
\bibitem{b2} Shead, Sam, ``Raspberry Pi delivery delays leave buyers hungry (and angry)", ZDNet, Octubre 2012.
\bibitem{b3} GSyC, ``Simple Network Managment Protocol,''  Universidad Rey Juan Carlos, 2013.
\bibitem{b4} LinkSprite, ``pcDuino1 Overview ,''  admin.
\bibitem{b5} ``OMAP3530 BeagleBoard" High perfomance and numerous expansion options; page 3, Dkc1.digikey.com , Mayo 2009.
\end{thebibliography}

\end{document}